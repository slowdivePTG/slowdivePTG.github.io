%%%%%%%%%%%%%%%%%%%%%%%%%%%%%%%%%%%%%%%%%
% Awesome Cover Letter
% XeLaTeX Template
% Version 1.1 (9/1/2016)
%
% This template has been downloaded from:
% http://www.LaTeXTemplates.com
%
% Original authors:
% Claud D. Park (posquit0.bj@gmail.com)
% Lars Richter (mail@ayeks.de)
% With modifications by:
% Vel (vel@latextemplates.com)
%
% License:
% CC BY-NC-SA 3.0 (http://creativecommons.org/licenses/by-nc-sa/3.0/)
%
% Important note:
% This template must be compiled with XeLaTeX, the below lines will ensure this
%!TEX TS-program = xelatex
%!TEX encoding = UTF-8 Unicode
%
%%%%%%%%%%%%%%%%%%%%%%%%%%%%%%%%%%%%%%%%%

%----------------------------------------------------------------------------------------
%	PACKAGES AND OTHER DOCUMENT CONFIGURATIONS
%----------------------------------------------------------------------------------------

\documentclass[11pt, a4paper]{awesome-cv} % A4 paper size by default, use 'letterpaper' for US letter

\geometry{left=2.54cm, top=1.5cm, right=2.54cm, bottom=2cm, footskip=.5cm} % Configure page margins with geometry
 
\fontdir[fonts/] % Specify the location of the included fonts

% Color for highlights
\colorlet{awesome}{awesome-red} % Default colors include: awesome-emerald, awesome-skyblue, awesome-red, awesome-pink, awesome-orange, awesome-nephritis, awesome-concrete, awesome-darknight
%\definecolor{awesome}{HTML}{CA63A8} % Uncomment if you would like to specify your own color

% Colors for text - uncomment and modify
%\definecolor{darktext}{HTML}{414141}
%\definecolor{text}{HTML}{414141}
%\definecolor{graytext}{HTML}{414141}
%\definecolor{lighttext}{HTML}{414141}

\renewcommand{\acvHeaderSocialSep}{\quad\textbar\quad} % If you would like to change the social information separator from a pipe (|) to something else

%----------------------------------------------------------------------------------------
%	PERSONAL INFORMATION
%	Comment any of the lines below if they are not required
%----------------------------------------------------------------------------------------

\name{Academic Statement of}{Purpose}
\position{Astronomy \& Astrophysics Ph.D.} 
\address{CHANG LIU\\U-M ID: 70531600}

\email{ptg.cliu@pku.edu.cn}
\homepage{https://slowdiveptg.github.io}
\github{slowdivePTG}
%\skype{skypeid}
%\stackoverflow{SOid}{SOname}
%\twitter{@twit}
%\linkedin{linkedin name}
%\reddit{reddit account}
%\xing{xing name}
%\ORCID{https://orcid.org/0000-0002-7866-4531} % Other text you want to include on this line

%\position{Undergraduate Student in Astrophysics} % Your expertise/fields
%\quote{``Explore the universe, benefit the society."} % A quote or statement

\makecvfooter{\today}{Chang Liu~~~·~~~State of Purpose}{\thepage} % Specify the letter footer with 3 arguments: (<left>, <center>, <right>), leave any of these blank if they are not needed
%----------------------------------------------------------------------------------------
%	RECIPIENT/POSITION/LETTER INFORMATION
%	All of the below lines must be filled out
%----------------------------------------------------------------------------------------

\recipient{Company Recruitment Team}{Initech Inc.\\4120 Freidrich Ln.\\Austin, TX 78744} % The company being applied to

\letterdate{\today} % The date on the letter, default is the date of compilation

\lettertitle{Job Application for Middle Manager} % The title of the letter

\letteropening{Dear Mr./Ms./Dr. LastName,} % How the letter is opened

\letterclosing{Sincerely,} % How the letter is closed

\letterenclosure[Attached]{Curriculum Vitae} % Any enclosures with the letter

\makecvfooter{\today}{Chang Liu~~~·~~~Statement of Purpose: University of Michigan}{} % Specify the letter footer with 3 arguments: (<left>, <center>, <right>), leave any of these blank if they are not needed
  
%----------------------------------------------------------------------------------------

\begin{document}

\makecvheader % Print the header

%\makelettertitle % Print the title

%----------------------------------------------------------------------------------------
%	LETTER CONTENT
%----------------------------------------------------------------------------------------

\begin{cvletter}

%------------------------------------------------

%%%%%%%%%%%%%%%%%%%%%%%%%%%%%%%%%%%%%%%%%
% Awesome Cover Letter
% XeLaTeX Template
% Version 1.1 (9/1/2016)
%
% This template has been downloaded from:
% http://www.LaTeXTemplates.com
%
% Original authors:
% Claud D. Park (posquit0.bj@gmail.com)
% Lars Richter (mail@ayeks.de)
% With modifications by:
% Vel (vel@latextemplates.com)
%
% License:
% CC BY-NC-SA 3.0 (http://creativecommons.org/licenses/by-nc-sa/3.0/)
%
% Important note:
% This template must be compiled with XeLaTeX, the below lines will ensure this
%!TEX TS-program = xelatex
%!TEX encoding = UTF-8 Unicode
%
%%%%%%%%%%%%%%%%%%%%%%%%%%%%%%%%%%%%%%%%%

%----------------------------------------------------------------------------------------
%	PACKAGES AND OTHER DOCUMENT CONFIGURATIONS
%----------------------------------------------------------------------------------------

\documentclass[11pt, a4paper]{awesome-cv} % A4 paper size by default, use 'letterpaper' for US letter

\geometry{left=2cm, top=1.5cm, right=2cm, bottom=2cm, footskip=.5cm} % Configure page margins with geometry
 
\fontdir[fonts/] % Specify the location of the included fonts

% Color for highlights
\colorlet{awesome}{awesome-red} % Default colors include: awesome-emerald, awesome-skyblue, awesome-red, awesome-pink, awesome-orange, awesome-nephritis, awesome-concrete, awesome-darknight
%\definecolor{awesome}{HTML}{CA63A8} % Uncomment if you would like to specify your own color

% Colors for text - uncomment and modify
%\definecolor{darktext}{HTML}{414141}
%\definecolor{text}{HTML}{414141}
%\definecolor{graytext}{HTML}{414141}
%\definecolor{lighttext}{HTML}{414141}

\renewcommand{\acvHeaderSocialSep}{\quad\textbar\quad} % If you would like to change the social information separator from a pipe (|) to something else

%----------------------------------------------------------------------------------------
%	PERSONAL INFORMATION
%	Comment any of the lines below if they are not required
%----------------------------------------------------------------------------------------

\name{Statement of Purpose}{}
\address{CHANG LIU\\Department of Astronomy, Peking University}

\email{ptg.cliu@pku.edu.cn}
\homepage{https://slowdiveptg.github.io}
\github{slowdivePTG}
%\skype{skypeid}
%\stackoverflow{SOid}{SOname}
%\twitter{@twit}
%\linkedin{linkedin name}
%\reddit{reddit account}
%\xing{xing name}
%\ORCID{https://orcid.org/0000-0002-7866-4531} % Other text you want to include on this line

%\position{Undergraduate Student in Astrophysics} % Your expertise/fields
%\quote{``Explore the universe, benefit the society."} % A quote or statement

\makecvfooter{\today}{Chang Liu~~~·~~~State of Purpose}{\thepage} % Specify the letter footer with 3 arguments: (<left>, <center>, <right>), leave any of these blank if they are not needed
%----------------------------------------------------------------------------------------
%	RECIPIENT/POSITION/LETTER INFORMATION
%	All of the below lines must be filled out
%----------------------------------------------------------------------------------------

\recipient{Company Recruitment Team}{Initech Inc.\\4120 Freidrich Ln.\\Austin, TX 78744} % The company being applied to

\letterdate{\today} % The date on the letter, default is the date of compilation

\lettertitle{Job Application for Middle Manager} % The title of the letter

\letteropening{Dear Mr./Ms./Dr. LastName,} % How the letter is opened

\letterclosing{Sincerely,} % How the letter is closed

\letterenclosure[Attached]{Curriculum Vitae} % Any enclosures with the letter

\makecvfooter{\today}{Chang Liu~~~·~~~Statement of Purpose}{} % Specify the letter footer with 3 arguments: (<left>, <center>, <right>), leave any of these blank if they are not needed
  
%----------------------------------------------------------------------------------------

\begin{document}

\makecvheader % Print the header

%\makelettertitle % Print the title

%----------------------------------------------------------------------------------------
%	LETTER CONTENT
%----------------------------------------------------------------------------------------

\begin{cvletter}

%------------------------------------------------

\lettersection{About Me}
As an undergraduate student in astrophysics, I have a strong urge of pursuing a Ph.D degree to continue my adventure on mystery outside our tiny planet. 

Though having been fascinated by astrophysics for long, the strong faith of being an astronomer, I believe, dates back in my freshman year in biology major. Though I got the \textbf{highest GPA (3.87/4)} among 120 freshmen in biology, I was finally tired of cumbersome taxonomy and test-tube-washing. Fortunate enough, after taking several fundamental courses on physics \& astronomy, I realized that for me, the subtle balance of astonishing physical pictures and beautiful mathematical structures is so well attained in astrophysics, when I decided to switch to astronomy major. The training in School of Physics at Peking University has then lain a firm foundation of \textit{mathematics} (calculus, linear algebra, PDE, and statistics) and \textit{physics} (analytical mechanics, statistical mechanics, electrodynamics, and quantum mechanics) for me. More advanced astrophysics courses (spectroscopy, cosmology, general relativity, gravitational-wave astrophysics, etc.) have equipped me for conducting research. Since my sophomore year, I have kept in \textbf{the first place in GPA (3.84/4)} among 28 students in astronomy major.

One of my greatest pursuits is to combine state-of-the-art observations with powerful computational methods to understand complex astrophysical environment. This sweet `temptation' has driven me to explore observational and computational astrophysics from various binary systems to astrochemistry \& astrobiology within interstellar medium.
%------------------------------------------------

\lettersection{Research Experience}

My first scientific project started at the end of my second year under the supervision of \textbf{Prof. Xian Chen} in \textbf{\textit{Kavli Institute for Astronomy and Astrophysics at Peking University}} and \textbf{Prof. Fujun Du} in \textbf{\textit{Purple Mountain Observatory, Chinese Academy of Sciences}}. Starting from a `crazy' implication of recent observations that the supermassive black hole in the Milky Way was active about 2-8 Myr ago, I worked as an `archaeologist' digging out the astrochemical history of our galaxy with numerical methods. During such an ancient AGN event, tremendous hard X-ray radiation should be able to efficiently penetrate the dusty Galactic disk and lead to the synthesis of complex molecules related to the origin of life. To run the long term astrochemical simulation, I built our chemical network based on a classic gas-phase model (\href{http://faculty.virginia.edu/ericherb/research_files/osu_01_2007}{\texttt{osu\_01\_2007}}) developed by OSU's astrophysical chemistry group. For self-consistency, I included necessary surface processes important for molecule formation. Under a detailed AGN and galactic absorption model, I carefully embedded X-ray ionization at various distances from the galactic center in the network. The simulation was executed with the \href{http://kromepackage.org}{\texttt{KROME}} package, a widely-used library-like code for astrophysical simulation with chemistry included. We are extremely excited to see that several typical prebiotic species show observable changes in abundance distribution in Galactic disk given former AGN events. Our paper is under final revision and will be submitted to ApJ soon.

My research was not limited to one single subfield. In the summer of 2019, I went to \textbf{\textit{Caltech}} as part of the Summer Undergraduate Research Fellowship (SURF) program and touched the field of transients for the first time under the mentorship of \textbf{Prof. Shrinivas Kulkarni}. With light curves from Zwicky Transient Facility (ZTF), a state-of-the-art optical time-domain survey with unprecedented field of view and survey efficiency, I conducted a systematic search for periodic white dwarf binaries. Starting with a population of 486,641 WD candidates identified in \textit{Gaia}, I performed a cross match between the so-far most complete catalog and ZTF data to select a subset of ~90,000 sources with over 100 observations in ZTF. My carefully designed periodogram based on Lomb-Scargle method was then applied to the extracted light curves. A sample of 81 periodic white dwarfs (WDs) with periods between 1 and 3 hr to our surprise stood out. With combined analysis of both shapes of light curves and color information from Gaia and PanSTARRS, I classified several sources of interest including a contact binary candidate with possibly the shortest period known and an unusual, strongly ellipsoidal-modulated double white dwarfs system with an extremely low-mass (ELM) component. A catalog of the 81 sources with our preliminary classification has been made. Our paper will be updated with continuing follow-ups before submission to ApJ.

After an attempt to reveal the nature of various stellar systems with degenerate components in observational way, I grow intensely interested in exploring more on the intrinsic physics of compact binaries. At present I am a visiting undergraduate student at \textbf{\textit{UC Santa Cruz}} working on my thesis on the mass transfer in compact binaries with \textbf{Prof. Enrico Ramirez-Ruiz}. We focus on the so-called AM CVn binary systems, of which the components are white dwarfs so close to each other that the mass transfer flow from the donor directly strikes the accretor. Gradually getting familiar with the numerical setup, I have been trying to conduct hydrodynamical simulations with the radiation MHD simulation code \href{http://flash.uchicago.edu/site/}{\texttt{FLASH}} to test the long-term stability of direct impact mass transfer. In the long run, we will try to work out how light curves and even gravitational radiation properties for AM CVn systems are constrained by model parameters with simulated data.

Conducting research in these most advanced facilities around the world enables me to I take a bite of various cutting-edge astronomical subfields, from astrochemistry to compact objects, and developed general skills of data analysis, visualization, and simulation on supercomputers. My proficiency in English communication and academic writing has risen to a new level after working overseas for months. More importantly, I gradually get to learn how to stay out of frustration, anxiety and loneliness, so that I am able to persist in a long-term project.
%------------------------------------------------

\end{cvletter}

%----------------------------------------------------------------------------------------

%\makeletterclosing % Print the signature and enclosures

\end{document}

\lettersection{Why U-M? - Academic Interests}

University of Michigan has been one of the leaders in astronomy since its foundation. Faculty at U-M work on a broad field of astrophysics and combine the latest observations with rapidly growing computational methods to answer the key questions in different areas. The group of extreme astrophysics makes use of world-class ground- \& space-based telescopes including Magellan, Chandra, and Swift, while the star \& planetary group is highly evolved in ALMA. Students at U-M are fortunate enough to be able to receive training as observers in MDM and Angell Hall Observatory. For graduate students at U-M, research are highly emphasized. Students are encouraged to work under the supervision of various faculty members before focusing on the thesis, which would enable me to explore the breadth of observational and computational astrophysics. Frequent oral presentations and academic writing training will definitely hone my skills as a rising astrophysicist.

I am not too surprised to find that University of Michigan meets so perfectly with my research experience. \textbf{Prof. Edwin Bergin} works on astrochemistry to understand planet formation and origin of biogenetic species with both data from ALMA and chemical simulations. My former simulation work on the genesis of prebiotic species in the Milky Way is closely related to his interests, and inspires me to go further in revealing the physical processes in protoplanetary disks under irradiation. I am also interested in \textbf{Prof. Emily Rauscher}’s 3-D simulations on exoplanets to find out how atmospheric circulation patterns on hot Jupiters are influenced by magnetic field and radiation. Besides, as I am also attracted by the nature of high energy transients, I do admire \textbf{Prof. Kayhan Gultekin}’s contribution on black holes and accretion processes, especially on determining the mass of black holes with observable properties of the accretion disk. \textbf{Prof. Jon Miller}’s work on probing the spins of black holes with X-ray properties is also what I would like to explore. My experiences on both observational and computational study on compact objects allow me to dive into the study on hydrodynamics and radiation around black holes.

I do appreciate it if you could consider my application. I am looking forward to meeting with you for a more detailed talk soon.
%------------------------------------------------

\end{cvletter}

%----------------------------------------------------------------------------------------

%\makeletterclosing % Print the signature and enclosures

\end{document}
%%%%%%%%%%%%%%%%%%%%%%%%%%%%%%%%%%%%%%%%%
% Awesome Cover Letter
% XeLaTeX Template
% Version 1.1 (9/1/2016)
%
% This template has been downloaded from:
% http://www.LaTeXTemplates.com
%
% Original authors:
% Claud D. Park (posquit0.bj@gmail.com)
% Lars Richter (mail@ayeks.de)
% With modifications by:
% Vel (vel@latextemplates.com)
%
% License:
% CC BY-NC-SA 3.0 (http://creativecommons.org/licenses/by-nc-sa/3.0/)
%
% Important note:
% This template must be compiled with XeLaTeX, the below lines will ensure this
%!TEX TS-program = xelatex
%!TEX encoding = UTF-8 Unicode
%
%%%%%%%%%%%%%%%%%%%%%%%%%%%%%%%%%%%%%%%%%

%----------------------------------------------------------------------------------------
%	PACKAGES AND OTHER DOCUMENT CONFIGURATIONS
%----------------------------------------------------------------------------------------

\documentclass[11pt, a4paper]{awesome-cv} % A4 paper size by default, use 'letterpaper' for US letter

\geometry{left=2cm, top=1.5cm, right=2cm, bottom=2cm, footskip=.5cm} % Configure page margins with geometry
 
\fontdir[fonts/] % Specify the location of the included fonts

% Color for highlights
\colorlet{awesome}{awesome-red} % Default colors include: awesome-emerald, awesome-skyblue, awesome-red, awesome-pink, awesome-orange, awesome-nephritis, awesome-concrete, awesome-darknight
%\definecolor{awesome}{HTML}{CA63A8} % Uncomment if you would like to specify your own color

% Colors for text - uncomment and modify
%\definecolor{darktext}{HTML}{414141}
%\definecolor{text}{HTML}{414141}
%\definecolor{graytext}{HTML}{414141}
%\definecolor{lighttext}{HTML}{414141}

\renewcommand{\acvHeaderSocialSep}{\quad\textbar\quad} % If you would like to change the social information separator from a pipe (|) to something else

%----------------------------------------------------------------------------------------
%	PERSONAL INFORMATION
%	Comment any of the lines below if they are not required
%----------------------------------------------------------------------------------------

\name{Personal Statement:}{MSU}
\address{CHANG LIU\\Department of Astronomy, Peking University}

\email{ptg.cliu@pku.edu.cn}
\homepage{https://slowdiveptg.github.io}
\github{slowdivePTG}
%\skype{skypeid}
%\stackoverflow{SOid}{SOname}
%\twitter{@twit}
%\linkedin{linkedin name}
%\reddit{reddit account}
%\xing{xing name}
%\ORCID{https://orcid.org/0000-0002-7866-4531} % Other text you want to include on this line

%\position{Undergraduate Student in Astrophysics} % Your expertise/fields
\quote{``Explore the universe, benefit the society."} % A quote or statement

\makecvfooter{\today}{Chang Liu~~~·~~~Personal Statement}{\thepage} % Specify the letter footer with 3 arguments: (<left>, <center>, <right>), leave any of these blank if they are not needed
%----------------------------------------------------------------------------------------
%	RECIPIENT/POSITION/LETTER INFORMATION
%	All of the below lines must be filled out
%----------------------------------------------------------------------------------------

\recipient{Company Recruitment Team}{Initech Inc.\\4120 Freidrich Ln.\\Austin, TX 78744} % The company being applied to

\letterdate{\today} % The date on the letter, default is the date of compilation

\lettertitle{Job Application for Middle Manager} % The title of the letter

\letteropening{Dear Mr./Ms./Dr. LastName,} % How the letter is opened

\letterclosing{Sincerely,} % How the letter is closed

\letterenclosure[Attached]{Curriculum Vitae} % Any enclosures with the letter

\makecvfooter{\today}{Chang Liu~~~·~~~Personal Statement: MSU}{} % Specify the letter footer with 3 arguments: (<left>, <center>, <right>), leave any of these blank if they are not needed
  
%----------------------------------------------------------------------------------------

\begin{document}

\makecvheader % Print the header

%\makelettertitle % Print the title

%----------------------------------------------------------------------------------------
%	LETTER CONTENT
%----------------------------------------------------------------------------------------

\begin{cvletter}

%------------------------------------------------

I was born and raised in a small town lying in a water-curved valley in South-east China. Historically, long before asphalt roads and automobiles came into being, mountains had kept my hometown marooned for thousands of years. Fortunately enough, in an era of rapid modernization, I was able to receive fundamental education before leaving for large cities as a first-generation college student. Still, my interests and character have been shaped in that valley.

My initial understanding of the mysterious world came from mountains covered by endless forests of bamboo where I used to search for hidden bamboo shoots in those rainy springs and stared at the unstained night sky without knowing any of the stars. Since then I have been fascinated by the magic of nature, under which everything changes in tune with time. There is little wonder that when I was old enough to comprehend the wonderful physical world on books, and later in my teens, the Internet, I became a fan of nature sciences, especially astronomy. Living in a region far from advanced educational resources, a great obstacle is lack of any teachers or fellows in astronomy. It was when I studied telescope structure and some basic astrophysics all by myself after finishing cumbersome coursework at senior high school that I realized my potential in self-learning and persistence on my beloved venture. Aiming at developing popular science, I also started an informal society of astronomy among my peers to conduct sidewalk astronomy practice.

This enthusiasm on astronomy has supported me for long, especially when I was inevitably in face with difficulties beyond my control in research. When I was conducting my simulation work on the chemical effect of an AGN, once I was stuck in modeling the chemical reactions under X-ray radiation. As complex organic molecules, which we studied, mainly form on dust grains, while our model only contains gaseous reactions. Unfortunately, neither my advisor (Prof. Xian Chen, Peking University) nor I was an expert in grain chemistry. I could not tolerate leaving my model self-inconsistent, so I traveled for hundreds of miles on train to visit one of our cooperators (Prof. Fujun Du, Purple Mountain Observatory). During the two-week stay in Purple Mountain Observatory, I completed the chemical network with necessary grain processes under the supervision of Prof. Du.

My root deeply embedded in mountains did not only inspired my academic pursuits, but also influenced my perspective on culture and diversity. Today, when shortage is only a remote and obscured nightmare in my hometown, all the ancient memories of isolation do not fade away. Our dialect lingers on. Xinchang Dialect, a variety of Wu Chinese, is used among no more than 400,000 individuals while sounds like nothing but chirp of birds for the rest billions of people in the country. Carrying a great amount of ancient Chinese pronunciation, vocabulary, and grammar, it is old as fossil and rich as a gold mine. However, it has long been considered demeaning to speak in our dialect in public. Over half of the new generation are even not taught this euphonious but difficult dialect at home. Staying in museums should not be the fate of Xinchang Dialect as well as all the fading dialects over the world. For years, I have been advertising for linguistic diversity. In high school, I systematically learned pronunciation and grammar of Xinchang Dialect through both oral and written literature as part of my extracurricular activities. Since I started college life in Beijing in a multicultral environment, I have been volunteered to introduce our dialect along with the culture behind to people from all over the country and the world. 

With intense interest and firm foundation in astronomy, I have strong urge to pursue a Ph.D degree in MSU to continue my adventure. With rich undergraduate research experience, I have gradually learned how to stay out of frustration, anxiety, and loneliness, so that I am able to persist in a long-term project. I would also contribute to the enviroment of diversity within MSU. Though my proficiency in English communication and academic writing has to be further honed, I am confident enough to guarantee my promising future as a graduate student in astronomy.
%------------------------------------------------

\end{cvletter}

%----------------------------------------------------------------------------------------

%\makeletterclosing % Print the signature and enclosures

\end{document}
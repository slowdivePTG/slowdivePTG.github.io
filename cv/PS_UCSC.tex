%%%%%%%%%%%%%%%%%%%%%%%%%%%%%%%%%%%%%%%%%
% Awesome Cover Letter
% XeLaTeX Template
% Version 1.1 (9/1/2016)
%
% This template has been downloaded from:
% http://www.LaTeXTemplates.com
%
% Original authors:
% Claud D. Park (posquit0.bj@gmail.com)
% Lars Richter (mail@ayeks.de)
% With modifications by:
% Vel (vel@latextemplates.com)
%
% License:
% CC BY-NC-SA 3.0 (http://creativecommons.org/licenses/by-nc-sa/3.0/)
%
% Important note:
% This template must be compiled with XeLaTeX, the below lines will ensure this
%!TEX TS-program = xelatex
%!TEX encoding = UTF-8 Unicode
%
%%%%%%%%%%%%%%%%%%%%%%%%%%%%%%%%%%%%%%%%%

%----------------------------------------------------------------------------------------
%	PACKAGES AND OTHER DOCUMENT CONFIGURATIONS
%----------------------------------------------------------------------------------------

\documentclass[11pt, a4paper]{awesome-cv} % A4 paper size by default, use 'letterpaper' for US letter

\geometry{left=2.54cm, top=1.5cm, right=2.54cm, bottom=2cm, footskip=.5cm} % Configure page margins with geometry
 
\fontdir[fonts/] % Specify the location of the included fonts

% Color for highlights
\colorlet{awesome}{awesome-red} % Default colors include: awesome-emerald, awesome-skyblue, awesome-red, awesome-pink, awesome-orange, awesome-nephritis, awesome-concrete, awesome-darknight
%\definecolor{awesome}{HTML}{CA63A8} % Uncomment if you would like to specify your own color

% Colors for text - uncomment and modify
%\definecolor{darktext}{HTML}{414141}
%\definecolor{text}{HTML}{414141}
%\definecolor{graytext}{HTML}{414141}
%\definecolor{lighttext}{HTML}{414141}

\renewcommand{\acvHeaderSocialSep}{\quad\textbar\quad} % If you would like to change the social information separator from a pipe (|) to something else

%----------------------------------------------------------------------------------------
%	PERSONAL INFORMATION
%	Comment any of the lines below if they are not required
%----------------------------------------------------------------------------------------

\name{Personal History Statement:}{UCSC}
\address{CHANG LIU\\Department of Astronomy, Peking University}

\email{ptg.cliu@pku.edu.cn}
\homepage{https://slowdiveptg.github.io}
\github{slowdivePTG}
%\skype{skypeid}
%\stackoverflow{SOid}{SOname}
%\twitter{@twit}
%\linkedin{linkedin name}
%\reddit{reddit account}
%\xing{xing name}
%\ORCID{https://orcid.org/0000-0002-7866-4531} % Other text you want to include on this line

%\position{Undergraduate Student in Astrophysics} % Your expertise/fields
\quote{``Explore the universe, benefit the society."} % A quote or statement

\makecvfooter{\today}{Chang Liu~~~·~~~Personal History Statement}{\thepage} % Specify the letter footer with 3 arguments: (<left>, <center>, <right>), leave any of these blank if they are not needed
%----------------------------------------------------------------------------------------
%	RECIPIENT/POSITION/LETTER INFORMATION
%	All of the below lines must be filled out
%----------------------------------------------------------------------------------------

\recipient{Graduate Student}{Caltech\\1200 E California Boulevard\\Pasadena, CA 91125} % The company being applied to

\letterdate{\today} % The date on the letter, default is the date of compilation

\lettertitle{Statement of Purpose} % The title of the letter

\letteropening{Dear Mr./Ms./Dr. LastName,} % How the letter is opened

\letterclosing{Sincerely,} % How the letter is closed

\letterenclosure[Attached]{Curriculum Vitae} % Any enclosures with the letter

\makecvfooter{\today}{Chang Liu~~~·~~~Personal History Statement: UCSC}{} % Specify the letter footer with 3 arguments: (<left>, <center>, <right>), leave any of these blank if they are not needed
  
%----------------------------------------------------------------------------------------

\begin{document}

\makecvheader % Print the header

%\makelettertitle % Print the title

%----------------------------------------------------------------------------------------
%	LETTER CONTENT
%----------------------------------------------------------------------------------------

\begin{cvletter}

%------------------------------------------------
I was born and raised in a small town, Xinchang (\textit{being prosper in a new era} in Chinese), lying in a water-carved valley in Southeast China. Historically, long before asphalt roads and automobiles came into being, mountains had kept my hometown marooned for thousands of years. Fortunately enough, in an era of rapid modernization, I was able to receive fundamental education before leaving for large cities as a first-generation college student. Still, life in that valley has shaped my interests and character.

\lettersection{Road to Astronomy}

My initial understanding of the mysterious world came from mountains covered by endless forests of bamboo where I used to search for hidden bamboo shoots in those rainy springs and stare at the yet unstained night sky without knowing any of the stars. Since then, I have been fascinated by the magic of nature, where everything seems to change in tune with time. There is little wonder that when I was old enough to comprehend the wonderful physical world on books, and later in my teens, the Internet, I became a fan of natural sciences, especially astronomy. 

I live in a region where teenagers in public high schools have little chance to obtain education resources in astronomy. In fact, in our province, which has a population of over 50 million, none of the universities can provide students with systematic astronomy education, not to mention public secondary schools. For years, I struggled with a lack of teachers and resources when learning astronomy, and it was not easy to find someone among my fellow students with similar interests. Under a stressful atmosphere of preparing for the college entrance examination, one of the few ways to step out of our narrow valley, an extracurricular hobby could be a luxury. It was when I studied telescope structure and some fundamental astrophysics online all by myself after finishing cumbersome coursework that I realized my potential in self-learning and persistence on my beloved venture. When everyone was dreaming of a promising yet unpredictable future but had no idea of what to become in the next decade, this hobby, as well as the faith to keep it, made me slightly different. Now, the same enthusiasm has driven me even further into the astronomical research.

But I did not stop, as I would always remember the famous saying by John Lennon, ``\textit{a dream you dream alone is only a dream}’’. It occurred to me that there could be other teenagers in my hometown who dream of revealing the principles of nature but rarely have access to teachers and friends just as I did, and I was the chosen one to get us to ``dream together’’. Aiming at developing \textbf{popular science}, I started an informal society of astronomy among my peers at high school to conduct sidewalk astronomy practice, including observing planets with telescopes and watching meteor showers such as the Perseids in our summer holidays. During then, I would share my knowledge of constellation classification and the related physics for some sources of interest. I have also tried to conduct popular science promotion online in literary essays as an attempt to combine ancient legends and stories of astronomy, the experience of an amateur astronomer, and the latest discovery in space.

\lettersection{Linguistic Diversity and More}

My roots deeply embedded in those mountains did not only inspired my academic pursuits but also influenced my perspective on culture and diversity. Today, when the history of shortage is only a remote and obscured nightmare in my hometown, all the ancient memories of isolation do not fade away. Our dialect lingers on. \textbf{Xinchang Dialect}, a variety of \href{https://en.wikipedia.org/wiki/Wu_Chinese}{\textbf{Wu Chinese}}, is still being used among no more than 400,000 individuals. While the dialect may sound like nothing but the chirps of birds for the rest billions of people in the country, it is part of our childhood, our familial bond, and our cultural history. Carrying a significant amount of ancient Chinese pronunciation, vocabulary, and grammar, it is old as fossil and invaluable as a gold mine. Our local drama Diao-Qiang, first performed over 400 years ago, is also based on the dialect. 

The popularization of Mandarin brings us education, business, and modern lifestyles, but at the same time hurts our dialect, at least to some degree. For long, speaking our dialect in public has been considered demeaning. Over half of the new generation born in the 21st century are not taught this euphonious but difficult language at home. It is also the case for many other dialects all around the world. When modernization brings health, wealth, and harmony to people, it is also sweeping off the diversity of the ways people speak, love, and understand each other.

Staying in museums should not be the fate of Xinchang Dialect as well as all the fading dialects over the world. The key is to remind society of the importance of keeping dialects as part of our life. For years, apart from persisting in communicating with my family in Xinchang Dialect, I have been advertising for \textbf{linguistic diversity}. In high school, I systematically learned the pronunciation and grammar of Xinchang Dialect through both oral and written literature as part of my extracurricular activities. Since I started college life in Beijing in a multicultural environment, I have volunteered to introduce our linguistic culture to people from all over the country and the world. 

\lettersection{Why UCSC - What I Can Do}

UCSC is a great place for me to continue my childhood dreams of becoming an astrophysicist. Staying at UCSC as a visitor for months, I am impressed by the friendly and diverse atmosphere here. I am sure that there are more I can do apart from my research. With the ambition of bringing astronomy to more teenagers, I could serve for the outreach events organized by Astronomy Club at UCSC as well as interpret cutting-edge astronomical literature in an everyday way. Besides, I would run for the preservation of all those ancient linguistic treasures to maintain the diversity of oral tradition for human-beings.

Thanks for considering my application. I am looking forward to meeting with you for a more detailed talk soon.
%------------------------------------------------

\end{cvletter}

%----------------------------------------------------------------------------------------

%\makeletterclosing % Print the signature and enclosures

\end{document}
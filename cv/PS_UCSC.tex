%%%%%%%%%%%%%%%%%%%%%%%%%%%%%%%%%%%%%%%%%
% Awesome Cover Letter
% XeLaTeX Template
% Version 1.1 (9/1/2016)
%
% This template has been downloaded from:
% http://www.LaTeXTemplates.com
%
% Original authors:
% Claud D. Park (posquit0.bj@gmail.com)
% Lars Richter (mail@ayeks.de)
% With modifications by:
% Vel (vel@latextemplates.com)
%
% License:
% CC BY-NC-SA 3.0 (http://creativecommons.org/licenses/by-nc-sa/3.0/)
%
% Important note:
% This template must be compiled with XeLaTeX, the below lines will ensure this
%!TEX TS-program = xelatex
%!TEX encoding = UTF-8 Unicode
%
%%%%%%%%%%%%%%%%%%%%%%%%%%%%%%%%%%%%%%%%%

%----------------------------------------------------------------------------------------
%	PACKAGES AND OTHER DOCUMENT CONFIGURATIONS
%----------------------------------------------------------------------------------------

\documentclass[11pt, a4paper]{awesome-cv} % A4 paper size by default, use 'letterpaper' for US letter

\geometry{left=2.54cm, top=1.5cm, right=2.54cm, bottom=2cm, footskip=.5cm} % Configure page margins with geometry
 
\fontdir[fonts/] % Specify the location of the included fonts

% Color for highlights
\colorlet{awesome}{awesome-red} % Default colors include: awesome-emerald, awesome-skyblue, awesome-red, awesome-pink, awesome-orange, awesome-nephritis, awesome-concrete, awesome-darknight
%\definecolor{awesome}{HTML}{CA63A8} % Uncomment if you would like to specify your own color

% Colors for text - uncomment and modify
%\definecolor{darktext}{HTML}{414141}
%\definecolor{text}{HTML}{414141}
%\definecolor{graytext}{HTML}{414141}
%\definecolor{lighttext}{HTML}{414141}

\renewcommand{\acvHeaderSocialSep}{\quad\textbar\quad} % If you would like to change the social information separator from a pipe (|) to something else

%----------------------------------------------------------------------------------------
%	PERSONAL INFORMATION
%	Comment any of the lines below if they are not required
%----------------------------------------------------------------------------------------

\name{Personal Statement:}{UCSC}
\address{CHANG LIU\\Department of Astronomy, Peking University}

\email{ptg.cliu@pku.edu.cn}
\homepage{https://slowdiveptg.github.io}
\github{slowdivePTG}
%\skype{skypeid}
%\stackoverflow{SOid}{SOname}
%\twitter{@twit}
%\linkedin{linkedin name}
%\reddit{reddit account}
%\xing{xing name}
%\ORCID{https://orcid.org/0000-0002-7866-4531} % Other text you want to include on this line

%\position{Undergraduate Student in Astrophysics} % Your expertise/fields
\quote{``Explore the universe, benefit the society."} % A quote or statement

\makecvfooter{\today}{Chang Liu~~~·~~~Personal Statement}{\thepage} % Specify the letter footer with 3 arguments: (<left>, <center>, <right>), leave any of these blank if they are not needed
%----------------------------------------------------------------------------------------
%	RECIPIENT/POSITION/LETTER INFORMATION
%	All of the below lines must be filled out
%----------------------------------------------------------------------------------------

\recipient{Graduate Student}{Caltech\\1200 E California Boulevard\\Pasadena, CA 91125} % The company being applied to

\letterdate{\today} % The date on the letter, default is the date of compilation

\lettertitle{Statement of Purpose} % The title of the letter

\letteropening{Dear Mr./Ms./Dr. LastName,} % How the letter is opened

\letterclosing{Sincerely,} % How the letter is closed

\letterenclosure[Attached]{Curriculum Vitae} % Any enclosures with the letter

\makecvfooter{\today}{Chang Liu~~~·~~~Personal Statement: UCSC}{} % Specify the letter footer with 3 arguments: (<left>, <center>, <right>), leave any of these blank if they are not needed
  
%----------------------------------------------------------------------------------------

\begin{document}

\makecvheader % Print the header

%\makelettertitle % Print the title

%----------------------------------------------------------------------------------------
%	LETTER CONTENT
%----------------------------------------------------------------------------------------

\begin{cvletter}

%------------------------------------------------
I was born and raised in a small town lying in a water-curved valley in Southeast China. Historically, long before asphalt roads and automobiles came into being, mountains had kept my hometown marooned for thousands of years. Fortunately enough, in an era of rapid modernization, I was able to receive fundamental education before leaving for large cities as a first-generation college student. Still, my interests and character have been shaped in that valley.

My initial understanding of the mysterious world came from mountains covered by endless forests of bamboo where I used to search for hidden bamboo shoots in those rainy springs and stare at the unstained night sky without knowing any of the stars. Since then, I have been fascinated by the magic of nature, under which everything changes in tune with time. There is little wonder that when I was old enough to comprehend the wonderful physical world on books, and later in my teens, the Internet, I became a fan of natural sciences, especially astronomy. Living in a region far from advanced educational resources, I suffered from a lack of teachers and fellows in astronomy. It was when I studied telescope structure and some basic astrophysics all by myself after finishing cumbersome coursework at senior high school that I realized my potential in self-learning and persistence on my beloved venture. Aiming at developing \textbf{popular science}, I also started an informal society of astronomy among my peers to conduct sidewalk astronomy practice including observing planets with telescopes, classification of constellations, and watching meteor showers. And now, this enthusiasm is driving me even further into the astronomical research.

My root deeply embedded in mountains did not only inspired my academic pursuits but also influenced my perspective on culture and diversity. Today, when shortage is only a remote and obscured nightmare in my hometown, all the ancient memories of isolation do not fade away. Our dialect lingers on. \textbf{Xinchang Dialect}, a variety of \href{https://en.wikipedia.org/wiki/Wu_Chinese}{\textbf{Wu Chinese}}, is used among no more than 400,000 individuals while it sounds like nothing but the chirp of birds for the rest billions of people in the country. Carrying a significant amount of ancient Chinese pronunciation, vocabulary, and grammar, it is old as fossil and rich as a gold mine. Our local drama Diao-Qiang, first performed over 400 years ago, is also based on the dialect. However, it has long been considered demeaning to speak in our dialect in public. Over half of the new generation are even not taught this euphonious but difficult language at home. Staying in museums should not be the fate of Xinchang Dialect as well as all the fading dialects over the world. For years, I have been advertising for \textbf{linguistic diversity}. In high school, I systematically learned the pronunciation and grammar of Xinchang Dialect through both oral and written literature as part of my extracurricular activities. Since I started college life in Beijing in a multicultural environment, I have been volunteered to introduce our dialect along with the culture behind to people from all over the country and the world. 

UCSC is a great destination for me to continue my childhood dreams of becoming an astrophysicist. Furthermore, I could serve in popular science groups to interpret cutting-edge astronomical literature in an everyday way. I would also run for the preservation of all those ancient linguistic treasures in order to maintain the diversity of oral tradition for human-beings.

%------------------------------------------------

\end{cvletter}

%----------------------------------------------------------------------------------------

%\makeletterclosing % Print the signature and enclosures

\end{document}
%%%%%%%%%%%%%%%%%%%%%%%%%%%%%%%%%%%%%%%%%
% Awesome Resume/CV
% XeLaTeX Template
% Version 1.2 (27/3/2017)
%
% This template has been downloaded from:
% http://www.LaTeXTemplates.com
%
% Original author:
% Claud D. Park (posquit0.bj@gmail.com) with modifications by 
% Vel (vel@latextemplates.com)
%
% License:
% CC BY-NC-SA 3.0 (http://creativecommons.org/licenses/by-nc-sa/3.0/)
%
% Important note:
% This template must be compiled with XeLaTeX, the below lines will ensure this
%!TEX TS-program = xelatex
%!TEX encoding = UTF-8 Unicode
%
%%%%%%%%%%%%%%%%%%%%%%%%%%%%%%%%%%%%%%%%%

%----------------------------------------------------------------------------------------
%	PACKAGES AND OTHER DOCUMENT CONFIGURATIONS
%----------------------------------------------------------------------------------------

\documentclass[11pt, a4paper]{awesome-cv} % A4 paper size by default, use 'letterpaper' for US letter
\geometry{left=2cm, top=1.5cm, right=2cm, bottom=2cm, footskip=.5cm} % Configure page margins with geometry

\fontdir[fonts/] % Specify the location of the included fonts

% Color for highlights
\colorlet{awesome}{awesome-red} % Default colors include: awesome-emerald, awesome-skyblue, awesome-red, awesome-pink, awesome-orange, awesome-nephritis, awesome-concrete, awesome-darknight
%\definecolor{awesome}{HTML}{CA63A8} % Uncomment if you would like to specify your own color

% Colors for text - uncomment and modify
%\definecolor{darktext}{HTML}{414141}
%\definecolor{text}{HTML}{414141}
%\definecolor{graytext}{HTML}{414141}
%\definecolor{lighttext}{HTML}{414141}

\renewcommand{\acvHeaderSocialSep}{\quad\textbar\quad} % If you would like to change the social information separator from a pipe (|) to something else

%----------------------------------------------------------------------------------------
%	PERSONAL INFORMATION
%	Comment any of the lines below if they are not required
%----------------------------------------------------------------------------------------

\name{Personal}{Statement}
\address{100871, Peking University, Haidian District, Beijing, China}

\email{ptg.cliu@pku.edu.cn}
\homepage{https://slowdiveptg.github.io}
\github{slowdivePTG}
%\skype{skypeid}
%\stackoverflow{SOid}{SOname}
%\twitter{@twit}
%\linkedin{linkedin name}
%\reddit{reddit account}
%\xing{xing name}
%\ORCID{https://orcid.org/0000-0002-7866-4531} % Other text you want to include on this line

\position{\color{gray}\fontsize{18pt}{1em}Chang Liu} % Your expertise/fields
\quote{``Explore the universe, benefit the society."} % A quote or statement

\makecvfooter{\today}{Chang Liu~~~·~~~Personal Statement}{\thepage} % Specify the letter footer with 3 arguments: (<left>, <center>, <right>), leave any of these blank if they are not needed

%----------------------------------------------------------------------------------------

\begin{document}

\makecvheader % Print the header
\cvsection{About Me}
\newline\soptextstyle
	I am Chang Liu and I have just ontained Bachelor in Science (with honours) at Department of Astronomy, School of Physics, Peking University.

	Though having been fascinated by astrophysics for long, my desire to become an astronomer, I believe, dates back in my freshman year in biology major. Though I got \textbf{the highest GPA (3.87/4)} among 120 freshmen, after taking several fundamental courses on physics \& astronomy, I realized that to me, the subtle balance of astonishing physical pictures and beautiful mathematical structures is so well attained in astrophysics that I decided to switch to astronomy major. The training in School of Physics at Peking University has lain a firm foundation of mathematics (calculus, linear algebra, PDE, and statistics) and physics (analytical mechanics, statistical mechanics, electrodynamics, and quantum mechanics) for me. Advanced courses in astrophysics (spectroscopy, cosmology, general relativity, gravitational-wave astrophysics, etc.) have equipped me for conducting research. since my sophomore year, I have kept in \textbf{the first place in GPA (3.83 / 4)} among 28 students in astronomy major.
	
	One of my greatest pursuits is to combine state-of-the-art observations with powerful computational methods to understand complex astrophysical environments. This sweet `temptation' has driven me to explore observational and computational astrophysics from astrochemistry within interstellar medium to various binary systems.

\cvsection{Research Experience}
\newline\soptextstyle

	My first scientific project started at the end of my second year under the supervision of \textbf{Prof. Xian Chen} in \textbf{\textit{Kavli institute for Astronomy and Astrophysics, Peking University}} and \textbf{Prof. Fujun Du} in \textbf{\textit{Purple Mountain Observatory, chinese Accdemy of Sciences}}. Starting from a `crazy' implication of recent observations that the supermassive blackhole in the Milky Way was active about 2-8 Myr ago, I worked as an `archaeologist' digging out the astrochemical history of our galaxy with numerical methods. During such an ancient AGN event, tremendous hard X-ray radiation could penetrate the dusty Galactic disk, causing the synthesis of complex molecules related to the origin of life. To run the long term astrochemical simulation, I built our chemical network based on a classic gas-phase model. For self-consistency, l included necessary surface processes important for molecule formation. Under a detailed AGN and galactic absorption model, I carefully embedded X-ray ionization at various distances from the galactic center in the network. The simulation was executed with the \texttt{KROME} package, a library-like code for astrophysical simulation with chemistry included. We are extremely excited to see that several typical prebiotic species show observable changes in abundance distribution in the Galactic disk given former AGN events. Our paper, of which lam the first author, has been submitted to ApJ.
	
	My research was not limited to one single subfield. In the summer of 2019, went to \textbf{\textit{Caltech}} as part of the Summer Undergraduate Research Fellowship (SURF) program and explored the field of transients for the first time under the mentorship of \textbf{Prof. Shri kulkarni}. With light curves from Zwicky Transient Facility (ZTF), a state-of-the-art optical time-domain survey with unprecedented field of view and survey efficiency, I conducted a systematic search for periodic white dwarfs (WDs). Starting with a population of 486,641 candidates identified in Gaia, I performed a cross match between the so-far most complete catalog and ZTF data, selecting a subset of \(\sim\)  90,000 sources with over 100 observations in ZTF. My carefully designed periodogram based on the Lomb-Scargle method was then applied to the extracted light curves. A sample of 81 periodic WDS with periods between 1 and 3 hr to our surprise stood out. With combined analysis of both shapes of light curves and color information from Gaia and PanSTARRS, I classified several sources of interest including a contact binary candidate with possibly the shortest period known and an unusual strongly ellipsoidal-modulated double WD system with an extremely low-mass (ELM) component. A catalog of the 81 sources with our preliminary classification has been made.
	
	After an attempt to reveal the nature of various stellar systems with degenerate components in observational way, I grow intensely interested in exploring the intrinsic physics of compact binaries. In the fall of 2019 I visited \textbf{\textit{UC Santa Cruz}} to work on my thesis on the mass transfer in compact binaries with \textbf{Prof. Enrico Ramirez-Ruiz}. We focus on the so-called direct impact systems, of which the components are WDs so close to each other that the mass transfer flow from the donor strikes the accretor as opposed to forming a disk. Understanding the nature of these systems, dominant sources of gravitational waves for space-based interferometers, will be of prime importance to next-generation detectors like LISA. To build an intuition of mass transfer, l independently built a 3-body integrator in Fortran to calculate the ballistic trajectory of a particle in Roche lobe overflow in a binary system, and successfully reproduced the trajectories and angular momenta transition processes in former literature. since then I have been conducting hydrodynamical simulations with the radiation MHD simulation code \texttt{FLASH} to test the long-term stability of direct impact mass transfer. With the Python package \texttt{yt}, I visualized the torque density to analyze how the orbital angular momentum evolves. In the long run, we will try to work out how light curves and even gravitational radiation properties for AM CVn systems are constrained by model parameters with simulated data.
	
	Conducting research in the most advanced facilities around the world enables me to sample various leading-edge subfields and develop general skills of data analysis, visualization, and simulation on supercomputers. My proficiency in English communication and academic writing has risen to a new level after working overseas for months. More importantly, I have gradually learned how to stay out of frustration, anxiety, and loneliness, so that I am able to persist in a long-term project.
%----------------------------------------------------------------------------------------

\end{document}
%%%%%%%%%%%%%%%%%%%%%%%%%%%%%%%%%%%%%%%%%
% Awesome Cover Letter
% XeLaTeX Template
% Version 1.1 (9/1/2016)
%
% This template has been downloaded from:
% http://www.LaTeXTemplates.com
%
% Original authors:
% Claud D. Park (posquit0.bj@gmail.com)
% Lars Richter (mail@ayeks.de)
% With modifications by:
% Vel (vel@latextemplates.com)
%
% License:
% CC BY-NC-SA 3.0 (http://creativecommons.org/licenses/by-nc-sa/3.0/)
%
% Important note:
% This template must be compiled with XeLaTeX, the below lines will ensure this
%!TEX TS-program = xelatex
%!TEX encoding = UTF-8 Unicode
%
%%%%%%%%%%%%%%%%%%%%%%%%%%%%%%%%%%%%%%%%%

%----------------------------------------------------------------------------------------
%	PACKAGES AND OTHER DOCUMENT CONFIGURATIONS
%----------------------------------------------------------------------------------------

\documentclass[11pt, a4paper]{awesome-cv} % A4 paper size by default, use 'letterpaper' for US letter

\geometry{left=2.54cm, top=2cm, right=2.54cm, bottom=2cm, footskip=.5cm} % Configure page margins with geometry
 
\fontdir[fonts/] % Specify the location of the included fonts

% Color for highlights
\colorlet{awesome}{awesome-red} % Default colors include: awesome-emerald, awesome-skyblue, awesome-red, awesome-pink, awesome-orange, awesome-nephritis, awesome-concrete, awesome-darknight
%\definecolor{awesome}{HTML}{CA63A8} % Uncomment if you would like to specify your own color

% Colors for text - uncomment and modify
%\definecolor{darktext}{HTML}{414141}
%\definecolor{text}{HTML}{414141}
%\definecolor{graytext}{HTML}{414141}
%\definecolor{lighttext}{HTML}{414141}

\renewcommand{\acvHeaderSocialSep}{\quad\textbar\quad} % If you would like to change the social information separator from a pipe (|) to something else

%----------------------------------------------------------------------------------------
%	PERSONAL INFORMATION
%	Comment any of the lines below if they are not required
%----------------------------------------------------------------------------------------

\name{Statement of Purpose:}{Harvard}
\address{CHANG LIU\\Department of Astronomy, Peking University}

\email{ptg.cliu@pku.edu.cn}
\homepage{https://slowdiveptg.github.io}
\github{slowdivePTG}
%\skype{skypeid}
%\stackoverflow{SOid}{SOname}
%\twitter{@twit}
%\linkedin{linkedin name}
%\reddit{reddit account}
%\xing{xing name}
%\ORCID{https://orcid.org/0000-0002-7866-4531} % Other text you want to include on this line

%\position{Undergraduate Student in Astrophysics} % Your expertise/fields
%\quote{``Explore the universe, benefit the society."} % A quote or statement

\makecvfooter{\today}{Chang Liu~~~·~~~State of Purpose}{\thepage} % Specify the letter footer with 3 arguments: (<left>, <center>, <right>), leave any of these blank if they are not needed
%----------------------------------------------------------------------------------------
%	RECIPIENT/POSITION/LETTER INFORMATION
%	All of the below lines must be filled out
%----------------------------------------------------------------------------------------

\recipient{Company Recruitment Team}{Initech Inc.\\4120 Freidrich Ln.\\Austin, TX 78744} % The company being applied to

\letterdate{\today} % The date on the letter, default is the date of compilation

\lettertitle{Job Application for Middle Manager} % The title of the letter

\letteropening{Dear Mr./Ms./Dr. LastName,} % How the letter is opened

\letterclosing{Sincerely,} % How the letter is closed

\letterenclosure[Attached]{Curriculum Vitae} % Any enclosures with the letter

\makecvfooter{\today}{Chang Liu~~~·~~~Statement of Purpose: Harvard}{} % Specify the letter footer with 3 arguments: (<left>, <center>, <right>), leave any of these blank if they are not needed
  
%----------------------------------------------------------------------------------------

\begin{document}

\makecvheader % Print the header

%\makelettertitle % Print the title

%----------------------------------------------------------------------------------------
%	LETTER CONTENT
%----------------------------------------------------------------------------------------

\begin{cvletter}

%------------------------------------------------

\lettersection{About Me}
As an undergraduate student in astrophysics, I have a strong urge to pursue a Ph.D degree  because to me, the subtle balance of astonishing physical pictures and beautiful mathematical structures is perfectly attained in astrophysics.
%Though having been fascinated by astrophysics for long, my desire to become an astronomer, I believe, dates back in my freshman year in biology major. Though I got the \textbf{highest GPA (3.87/4)} among 120 freshmen, after taking several fundamental courses on physics \& astronomy, I realized that to me, the subtle balance of astonishing physical pictures and beautiful mathematical structures is so well attained in astrophysics that I decided to switch to astronomy major. The training in School of Physics at Peking University has lain a firm foundation of \textit{mathematics} (calculus, linear algebra, PDE, and statistics) and \textit{physics} (analytical mechanics, statistical mechanics, electrodynamics, and quantum mechanics) for me. Advanced courses in astrophysics (spectroscopy, cosmology, general relativity, gravitational-wave astrophysics, etc.) have equipped me for conducting research. Since my sophomore year, I have kept in \textbf{the first place in GPA (3.84/4)} among 28 students in astronomy major.

One of my greatest pursuits is to combine state-of-the-art observations with powerful computational methods to understand complex astrophysical environments. This sweet `temptation' has driven me to explore observational and computational astrophysics from astrochemistry within interstellar medium to various binary systems.

%------------------------------------------------

\lettersection{Research Experience}

My first scientific project started at the end of my second year under the supervision of \textbf{Prof. Xian Chen} in \textbf{\textit{Kavli Institute for Astronomy and Astrophysics, Peking University}} and \textbf{Prof. Fujun Du} in \textbf{\textit{Purple Mountain Observatory, Chinese Academy of Sciences}}. Starting from a `crazy' implication of recent observations that the supermassive black hole in the Milky Way was active about 2-8 Myr ago, when tremendous hard X-ray could penetrate the dusty Galactic disk to cause the synthesis of complex molecules related to the origin of life, I worked as an `archaeologist' digging out the astrochemical history of our galaxy. To run the long term astrochemical simulation, I built our chemical network based on a classic gas-phase model (\href{http://faculty.virginia.edu/ericherb/research_files/osu_01_2007}{\texttt{osu\_01\_2007}}). For self-consistency, I included necessary surface processes important for molecule formation. Under a detailed AGN and galactic absorption model, I carefully embedded X-ray ionization at various distances from the galactic center in the network. With the simulation executed by the \href{http://kromepackage.org}{\texttt{KROME}} package, we are extremely excited to see that several typical prebiotic species show observable changes in Galactic distribution given former AGN events. Our paper, of which I am the first author, is under final revision and will be submitted to ApJ within the next month.

My research was not limited to one single subfield. In the summer of 2019, I went to \textbf{\textit{Caltech}} as part of the Summer Undergraduate Research Fellowship (SURF) program and explored the field of transients for the first time under the mentorship of \textbf{Prof. Shri Kulkarni}. With light curves from Zwicky Transient Facility (ZTF), a state-of-the-art optical time-domain survey with unprecedented field of view and survey efficiency, I conducted a systematic search for periodic white dwarfs (WDs). I performed a cross match between the so-far most complete catalog by \textit{Gaia} and ZTF, selecting a subset of $\sim$90,000 sources with rich ZTF data. My carefully designed periodogram based on the Lomb-Scargle method was then applied to the extracted light curves. A sample of 81 periodic WDs with periods between 1 and 3 hr stood out. With combined analysis of both shapes of light curves and color information from Gaia and PanSTARRS, I classified several unusual sources including a contact binary candidate with possibly the shortest period known and a strongly ellipsoidal-modulated double WD system with an extremely low-mass (ELM) component. Our paper will be updated with continuing follow-ups before submission to ApJ.

After working on various stellar systems with degenerate components in observational way, I grow intensely interested in the intrinsic physics of compact binaries. At present I am a visiting undergraduate student at \textbf{\textit{UC Santa Cruz}} working on my thesis on the mass transfer in compact binaries with \textbf{Prof. Enrico Ramirez-Ruiz}. We focus on the direct impact systems, of which the components are WDs so close to each other that the mass transfer flow strikes the accretor as opposed to forming a disk. They dominate gravitational waves sources in the band of space-based interferometers (like \textit{LISA}). To gain an intuition, I independently built a 3-body integrator in Fortran to calculate the ballistic trajectory of a particle in Roche lobe overflow, and successfully reproduced the trajectories and angular momenta transition in literature. Since then I have been conducting hydrodynamical simulations with the radiation MHD simulation code \href{http://flash.uchicago.edu/site/}{\texttt{FLASH}} to test the long-term stability of direct impact mass transfer. With the Python package \href{https://yt-project.org}{\texttt{yt}}, I visualized the torque density to analyze how the orbital angular momentum evolves. In the long run, we will try to predict electromagnetic and gravitational radiation properties of these systems given model parameters.

Conducting research in the most advanced facilities around the world enables me to sample leading-edge subfields and develop skills of data analysis, visualization, and simulations. My proficiency in English has risen to a new level after working overseas for months. More importantly, I have gradually learned how to stay out of frustration, anxiety, and loneliness, in order to persist in a long-term project.
%------------------------------------------------

\lettersection{Why Harvard? - Academic Interests}

With unparalleled observational resources and computational power, Harvard is one of the world’s centers for astronomy. The breadth of research in the Harvard-Smithsonian Center for Astrophysics makes interdisciplinarity possible. Astrophysicists, planetologists, and biologists at Harvard have been working together to understand how life originates and evolves on a planet. In the Institute for Theory and Computation (ITC), the cooperation between computer scientists and theorists motivates the next-generation telescopes, such as GMT and JWST. It is little wonder that graduate students at Harvard have access to diverse cutting-edge astrophysics knowledge on courses as well as seminars and conferences. 

I am not too surprised to find that Harvard meets so perfectly with my research. \textbf{Prof. Avi Loeb} covers a variety of astrophysical objects. He has been working on extraterrestrial life and has recently studied evaporation effect of an AGN to planetary atmospheres nearby. I would really like to further explore the so-called galactic habitable zone with the help of astrochemical simulations taking supernovae and even galactic mergers into account. \textbf{Prof. Karin Öberg}’s work on complex molecules in molecular clouds and protoplanetary disks, which combines experimental, observational, and computational astrochemistry, is also exciting. I appreciate the recent work within \textbf{Prof. Selma de Mink}’s group on the roles binaries are playing in the formation of massive stars, core-collapse supernovae explosions, and even Cosmic Reionization. Besides, I do admire \textbf{Prof. Lars Hernquist}’s work on stellar hydrodynamical simulations with \texttt{Illustris}.

I do appreciate it if you could consider my application. I am looking forward to meeting with you for a more detailed talk soon.
%------------------------------------------------

\end{cvletter}

%----------------------------------------------------------------------------------------

%\makeletterclosing % Print the signature and enclosures

\end{document}
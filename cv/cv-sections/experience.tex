%----------------------------------------------------------------------------------------
%	SECTION TITLE
%----------------------------------------------------------------------------------------

\cvsection{Research Experience}

%----------------------------------------------------------------------------------------
%	SECTION CONTENT
%----------------------------------------------------------------------------------------

\begin{cvexperiences}

%------------------------------------------------

\cvexperience
{Impact of an Active Sgr A* on the Synthesis of Molecular Species Throughout the Milky Way} % Job title
{Department of Astronomy, Peking University} % Organization
{Beijing, China} % Location
{Jul 2018 - Nov 2019} % Date(s)
{Mentors: Xian Chen \& Fujun Du}
{ % Description(s) of tasks/responsibilities
\begin{cvitems}
\item {\textit{Undergraduate Research \& Training Program - National Innovation Training Program}}
\item {Investigated the impacts of an AGN on the synthesis of prebiotic/organic molecules to indicate the potential correlation between an active supermassive black hole and both the origin and the evolution of life.}
\item {Calculated the ionization rates of electromagnetic radiation caused by accretion of the supermassive black hole in the Milky Way with Galactic absorption considered.}
\item {Completed the classic gas-phase network \href{http://faculty.virginia.edu/ericherb/research_files/osu_01_2007}{\texttt{osu\_01\_2007}} by adding  X-ray ionization and necessary grain processes important for synthesis of complex species.}
\item {Simulated the chemical evolution of crucial precursors for interstellar prebiotic molecules with \href{http://kromepackage.org}{\texttt{KROME}}, with the discovery of observable change in distribution for important molecules.}
\end{cvitems}
}

\cvexperience
{A Systematic Search For Periodic White Dwarfs Using ZTF Data} % Job title
{Astronomy Department, Caltech} % Organization
{Pasadena, US} % Location
{Jun 2019 - Aug 2019} % Date(s)
{Mentor: Shrinivas R. Kulkarni}
{ % Description(s) of tasks/responsibilities
	\begin{cvitems}
		\item {\textit{Summer Undergraduate Research Fellowship (SURF)}}
		\item {Explored the potential of the state-of-the-art time-domain facility - Zwicky Transient Facility (ZTF) by conducting a systematic search for periodic white dwarfs with periods lying within 1-3 hr.}
		\item {Conducted a cross match between \textit{Gaia} and ZTF, selecting $\sim$ 90,000 \textit{Gaia} sources with enough ZTF records.}
		\item {A number of 81 sources stood out as periodic under a well-designed periodogram based on Lomb-Scargle method.}
		\item {Analyzed the shapes of light curves derived from ZTF as well as color information from \textit{Gaia} and PanSTARRS.}
		\item {Discovered various sources of interest including an unusual strongly ellipsoidal-modulated double white dwarfs system with an extremely low-mass (ELM) component.}
	\end{cvitems}
}

\cvexperience
{The Hydrodynamics of Binary Mass Transfer in Compact Binaries} % Job title
{Department of Astronomy and Astrophysics, UC Santa Cruz} % Organization
{Santa Cruz, US} % Location
{Oct 2019 - Jun 2020} % Date(s)
{Mentor: Enrico Ramirez-Ruiz}
{ % Description(s) of tasks/responsibilities
	\begin{cvitems}
		\item {\textit{Undergraduate thesis}}
		\item {Studied the stability of mass transfer in a direct impact mass transfer white dwarf binary with hydrodynamical simulation.}
		\item {Built a 3-body integrator in Fortran to calculate the ballistic trajectory of a particle in Roche lobe overflow in a binary system.}
		\item {For the very first time, shed light on a fully hydrodynamical understanding on the rather complicated ultra-compact binary systems.}
		\item {Visualized the feedback of torques of the accreted materials on the orbital evolution of double white dwarfs with the open source Python package \href{https://yt-project.org}{\texttt{yt}}.}
	\end{cvitems}
}

\cvexperience
{A Search for Stellar-Mass Black Holes in Microlensing Events} % Job title
{The Kavli Instititute for Astronomy and Astrophysics, Peking University} % Organization
{Beijing, China} % Location
{Sep 2020 - Aug 2021} % Date(s)
{Mentor: Subo Dong}
{ % Description(s) of tasks/responsibilities
	\begin{cvitems}
		\item {\textit{Research Assistant}}
		\item {Built a pipeline which automatically select microlensing candidates in Gaia Alerts and gather prerequisite knowledge for follow-up observations.}
		\item {Monitored and modeled several long-timescale microlensing candidates.}
		\item {Estimated the event rate of stellar-mass black holes based on an analytical Galactic model.}
	\end{cvitems}
}

\end{cvexperiences}
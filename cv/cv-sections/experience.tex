%----------------------------------------------------------------------------------------
%	SECTION TITLE
%----------------------------------------------------------------------------------------

\cvsection{Experience}

%----------------------------------------------------------------------------------------
%	SECTION CONTENT
%----------------------------------------------------------------------------------------

\begin{cvexperiences}

%------------------------------------------------

\cvexperience
{Impact of an Active Sgr A* on the Synthesis of Prebiotic Molecules Throughout the Milky Way} % Job title
{Department of Astronomy, Peking University} % Organization
{Beijing, China} % Location
{Jul 2018 - Nov 2019} % Date(s)
{Mentors: Xian Chen \& Fujun Du}
{ % Description(s) of tasks/responsibilities
\begin{cvitems}
\item {\textit{Undergraduate Research \& Training Program - National Innovation Training Program}}
\item {Investigated the impacts of an AGN on the synthesis of prebiotic/organic molecules to indicate the potential correlation between an active supermassive black hole and both the origin and the evolution of life.}
\item {Calculated the ionization rates of electromagnetic radiation caused by accretion of the supermassive black hole in the Milky Way with Galactic absorption considered.}
\item {Completed the classic gas-phase network \href{http://faculty.virginia.edu/ericherb/research_files/osu_01_2007}{\texttt{osu\_01\_2007}} by adding  X-ray ionization and necessary grain processes important for synthesis of complex species.}
\item {Simulated the chemical evolution of crucial precursors for interstellar prebiotic molecules with \href{http://kromepackage.org}{\texttt{KROME}}.}
\end{cvitems}
}

\cvexperience
{A Systematic Search For Periodic White Dwarfs Using ZTF Data} % Job title
{Astronomy Department, Caltech} % Organization
{Pasadena, US} % Location
{Jun 2019 - Aug 2019} % Date(s)
{Mentor: Shrinivas R. Kulkarni}
{ % Description(s) of tasks/responsibilities
	\begin{cvitems}
		\item {\textit{Summer Undergraduate Research Fellowship (SURF)}}
		\item {Explored the potential of the state-of-the-art time-domain facility - Zwicky Transient Facility (ZTF) by conducting a systematic search for periodic white dwarfs with periods lying within 1-3 hr.}
		\item {Conducted a cross match between \textit{Gaia} and ZTF was performed, selecting $\sim$ 90,000 \textit{Gaia} sources with enough ZTF records.}
		\item {A number of 81 sources stood out as periodic under a well-designed periodogram based on Lomb-Scargle method.}
		\item {Analyzed the shapes of light curves derived from ZTF as well as color analysis with \textit{Gaia} and PanSTARRS.}
		\item {Discovered various sources of interest including an unusual strongly ellipsoidal-modulated double white dwarfs system with an extremely low-mass (ELM) component.}
		\item {Made a catalog of the 81 sources with our preliminary classification.}
	\end{cvitems}
}

\cvexperience
{The Hydrodynamics of Binary Mass Transfer in Compact Binaries} % Job title
{Department of Astronomy and Astrophysics, UC Santa Cruz} % Organization
{Santa Cruz, US} % Location
{Oct 2019 -} % Date(s)
{Mentor: Enrico Ramirez-Ruiz}
{ % Description(s) of tasks/responsibilities
	\begin{cvitems}
		\item {\textit{Undergraduate thesis}}
		\item {Study the stability of mass transfer in a Direct Impact mass transfer white dwarf binary with hydrodynamical simulation.}
		\item {Focus on the feedback of torques of the accreted materials on the orbital evolution of double white dwarfs.}
		\item {Simulations are executed with the radiation MHD simulation code, \href{http://flash.uchicago.edu/site/}{\texttt{FLASH}}.}
	\end{cvitems}
}

\end{cvexperiences}